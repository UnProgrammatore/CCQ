%************************************************
\section{Risultati}
\label{sec:risultati}
%************************************************


I risultati che abbiamo non sono ancora ottimali per trarre conclusioni precise in merito alle prestazioni ed allo speed-up.
Per ora siamo riusciti a fattorizzare numeri sotto le 50 cifre su un numero massimo di nodi pari a 8, i numeri con una quantità superiore di cifre hanno causato l'uccisione del processo dal gestore di code di Galileo, il supercomputer su cui abbiamo eseguito i calcoli, a tal proposito si vedano gli sviluppo futuri per quest'algoritmo nell'ottica di creare un sistema di checkpoint per non perdere il calcolo fatto qualora si eccedesse il tempo a propria disposizione. \\
Verranno presentati di seguito alcuni dei risultati significativi.\\ \\

\emph{Fattorizzazione di un numero a 30 cifre} \\ \\
\begin{tabular}{|r|ll|r|} 
Numero di processori & Tempo crivello & Tempo eliminazione di Gauss & Tempo totale \\
2 & 2.357121 & 3.806429 & 6.367990 \\
4 & 1.891235 & 3.797348 & 5.896448 \\
8 & 1.923282 & 1.788766 & 3.935302
\end{tabular}
\\ \\ \\

\emph{Fattorizzazione di un numero a 40 cifre} \\ \\
\begin{tabular}{|r|ll|r|} 
Numero di processori & Tempo crivello & Tempo eliminazione di Gauss & Tempo totale \\
2 & 320.353972 & 250.406708 & 573.848776 \\
4 & 202.583528 & 276.894561 & 482.333897
\end{tabular}
\\ \\ \\

\emph{Fattorizzazione di un numero a 50 cifre} \\ \\
\begin{tabular}{|r|ll|r|} 
Numero di processori & Tempo crivello & Tempo eliminazione di Gauss & Tempo totale \\
4 & 7683.857860 & 5171.678976 & 12884.067465 \\
8 & 3956.152933 & 5621.893790 & 9603.754040
\end{tabular}
\\ \\ \\

Per valutare lo speed-up dell'algoritmo verranno presi in considerazione i tempi del crivello, la parte parallelizzata (e l'unica parallelizzabile) dell'algoritmo.
Notiamo che nel caso del numero di 30 cifre non vi è un vero e proprio speed-up lineare, abbiamo anzi un aumento dei tempi di crivello. Questo è attribuibile al fatto che, con un numero così piccolo per l'algoritmo, l'overhead generato dall'aumento del numero di nodi anche ridotto è sufficiente a vanificare i miglioramenti ottenuti dall'aumento del potenziale di calcolo.
Nel caso del numero a 40 cifre possiamo apprezzare uno speed-up leggermente migliore.
Infine, nel caso del numero a 50 cifre, lo speed-up ottenuto è quasi lineare.
Infatti, all'aumentare del numero di cifre, l'overhead dovuto all'aumentare del numero dei nodi viene meglio ammortizzato.