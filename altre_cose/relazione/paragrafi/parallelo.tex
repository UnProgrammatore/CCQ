%************************************************
\section{Algoritmo parallelizzato}
\label{sec:parallelo}
%************************************************
Descriveremo l'architettura parallela dell'algoritmo.
\subsection{Base di fattori}
La prima parte dell'algoritmo consiste nella ricerca di una base di
fattori opportuna. Il nodo master potrebbe ricercare la base ed inviare il
vettore di primi che la compongono ai restanti nodi slave del
cluster. Siccome gli slave resterebbero inattivi fino alla ricezione
della base, in questa implementazione parallela si è deciso di far
determinare la base di fattori ad ogni nodo indipendentemente. Tutti i
nodi impiegheranno un tempo circa uguale per determinarla e inizieranno
subito la parte successiva dell'algoritmo evitando il trasferimento di
questi dati.
\subsection{Il ruolo del nodo master}
Il nodo master entra in una procedura che colleziona i risultati
provenienti dagli slave durante la fase di crivello. Ricordiamo che in
questa parte si deve ricercare per quali valori di $A$ il polinomio $Q(A)$
si fattorizza completamente sulla base di fattori. I nodi effettuano
dunque le fattorizzazioni in simultanea ed
inviano il vettore degli esponenti e il vettori degli $(A + s)$
ricavati. Sono necessarie $K+10$ fattorizzazioni per proseguire
nell'algoritmo (dove $K$ è la dimensione della base di fattori) dunque
una volta collezionato tale numero di risultati, il master interrompe
l'ascolto dei messaggi MPI in arrivo dagli slave ed invia un segnale
di stop a tutti indicando di concludere la procedura di crivello.

Terminata la parte di "accumulo dei risultati" la parte parallela
dell'algortimo finisce e l'esecuzione riprende come nella versione
seriale. La parte di eliminazione gaussiana e quella finale di calcolo
delle congruenze non è infatti parallelizzabile.

In dettaglio, estratto dal file \texttt{quadratic\char`_sieve.c}, la procedura
di ricezione del master:
\begin{lstlisting}
while(fact_count < max_fact + base_dim) {
    /* Ricevo il vettore di esponenti */
    MPI_Recv(buffer_exp, base_dim, MPI_UNSIGNED,
	     MPI_ANY_SOURCE, ROW_TAG, 
	     MPI_COMM_WORLD, &status);
    source = status.MPI_SOURCE;
    /* Salvo il contenuto del buffer */
    for(unsigned int i = 0; i < base_dim; ++i) 
      set_matrix(exponents, fact_count, i, buffer_exp[i]);
    /* Ricevo l'mpz contenente (A + s) */
    MPI_Recv(buffer_As, BUFFER_DIM, MPI_UNSIGNED_CHAR, source, 
	     AS_TAG, MPI_COMM_WORLD, &status);
    /* Estrapolo la dimensione del vettore mpz */
    MPI_Get_count(&status, MPI_UNSIGNED_CHAR, &count);
    /* Lo "importo" mediante questa primitiva della libreria */
    mpz_import(As[fact_count], count, 1, 1, 1, 0, buffer_As);
    ++fact_count;
}
\end{lstlisting}
Sono presenti due tag per le chiamte a \texttt{MPI\char`_Recv} per forzare la
ricezione in sequenza di un vettore di esponenti, il \texttt{ROW\char`_TAG}, ed un
vettore di $(A + s)$, l'\texttt{AS\char`_TAG}. Viene estratto l'id del mittente dalla
ricezione del vettore degli esponenti. Questo è necessario per
prelevare il vettore degli $(A + s)$ relativo agli stessi dati. Altri
vettori $(A + s)$ inviati da altri slave potrebbero essere giunti
prima nella coda di ricezione rispetto ai messaggi dello slave
di cui stiamo considerando il vettore degli esponenti. \'E
fondamentale l'associazione vettore 
esponenti e vettore $(A + s)$.

Sorge tuttavia una problematica di uso efficiente delle risorse. Il
nodo master resta in attesa delle fattorizzazioni durante tutta la fase di
crivello, per numeri sufficientemente grossi i polinomi $Q(A)$ che si
fattorizzano completamente sono più "rari", lasciando quindi il processore del nodo
inattivo per la maggior parte del tempo. Una soluzione a questo
problema che non preveda alcuna modifica al codice è la seguente:
si lanciano i processi mpi e si istruisce il gestore delle code
del cluster ad attivare un singolo processo 
per nodo ad eccezione del master. Su questo sarà infatti lanciato un
secondo processo mpi che occuperà i tempi di inattvità della cpu
dovuti al primo processo in attesa, ricercando anch'esso
fattorizzazioni per i $Q(A)$. Nell'implementazione dell'algoritmo
proposta, questa soluzione non è
attuabile con  successo poichè \texttt{MPI\char`_Recv} effettua un attesa attiva
consumando comunque il 100\% della risorsa cpu. Si potrebbe sostituire
la \texttt{MPI\char`_Recv} con la ricezione asincrona e mandare in sleep il processo
per un lasso di tempo fissato forzando un attessa che, seppur sempre
attiva, avvenga con frequenza inferiore.
\subsection{Il ruolo dei nodi slave}
Il crivello è la parte più onerosa di tutto l'algoritmo. Viene
richiesto di provare svariati valori di $A$ per tentare di fattorizzare
completamente il polinomio $Q(A)$ calcolato. Consideriamo un
insieme di valore di $A$ che è l'intervallo $[0, M]$ e decomponiamolo in
tante parti quanti sono i nodi slave del cluster. Lo slave i-mo
effettuerà i calcoli sui valori di $A$ nell'intervallo
$[B_i, F_i]$ grande $M/\mathrm{n\char`_slaves}$. 
Nella versione seriale dalla procedura di crivello, si 
ritornavano due strutture dati contenenti i vettori degli esponenti
delle fattorizzazioni e il vettore degli $(A + s)$. Nella versione
parallela memorizzeremo temporaneamente queste "soluzioni" per poi
inviarle al master, tramite MPI, ogni volta che si saranno tentati $n$ valori
di $A$. Il comportamento della funzione di crivello parallelo è identico
alla controparte parallela con la differenza che anzichè salvare i risutlati
in una struttura dati si inviano mediante il protocollo MPI al nodo
master. L'architettura del nostro sistema è però ibrida: abbiamo svariati
processi MPI al cui interno girano svariati thread. Analizzeremo più nel
dettaglio come sarà decomposto il dominio dei valori di $A$ per
assegnare un pezzo a ciascun processo e, al loro interno, a ciascun
thread.
\subsection{Decomposizione del dominio}
La prima suddivisione del dominio: a livello di processo MPI.
\begin{lstlisting}
mpz_set_ui(begin, interval * (rank - 1));
do {
   stop_flag = smart_sieve(N, factor_base, n_primes, solutions,
                	  begin, interval,
		  	  block_size, max_fact);

   mpz_add_ui(begin, begin, interval * (comm_size-1));
} while(!stop_flag);
\end{lstlisting}
Supponiamo di considerare lo slave i-mo. Il rank di tale slave sarà
$i+1$ (per via della presenza del master). Il parametro "interval"
rappresenta il blocco di valori di $A$ che ogni slave utilizza alla
volta. Lo slave i-mo inizialmente autodetermina in base al suo
rank l'intervallo sul quale deve operare assegnando alla variabile
"begin" il valore "interval * i" (si ricorda che $rank = i + 1$).
La funzione \texttt{smart\char`_sieve} opera dunque sull'intervallo $[begin, begin +
  interval]$. Se nessun segnale di stop è arrivato dal master lui
autodetermina l'intervallo successivo da esaminare tenendo conto della
presenza degli altri nodi facendo \texttt{begin = begin + (interval *
comm\char`_size - 1)}. La variabile \texttt{comm\char`_size} contiene la dimensione del
cluster alla quale va tolto il nodo master che non esegue il crivello.

Passiamo ora alla seconda suddivisione: a livello di thread.
\begin{lstlisting}
/* Inizio della parte di codice eseguita da ogni thread */
#pragma omp parallel
{ 
  /* ... */
 
  int threads = omp_get_num_threads();
  int thread_id = omp_get_thread_num();
  unsigned int dom_decomp = interval / (threads);
  /* begin_thread = begin + (dom_decomp * thread_id) */
  mpz_add_ui(begin_thread, begin, dom_decomp * thread_id); 
  /* end_thread = begin_thread + dom_decomp */
  mpz_add_ui(end_thread, begin_thread, dom_decomp); 

  /* Parte d crivello */
  /* ... */
}
\end{lstlisting}
Similmente a come avveniva a livello di processi MPI, l'intervallo di
dati (parametro "interval") sul quale viene chiamato
\texttt{smart\char`_sieve} viene suddiviso in tanti pezzi quanti i thread
disponibili nel processore.
\subsection{Trasmissione dati al master}
\begin{lstlisting}
for(i = 0; i < block_size; ++i)
  /* Se Q(A) e' stato fattorizzato completamente */
  if(mpz_cmp_ui(evaluated_poly[i], 1) == 0) {
    ++fact_count;
    for(k = 0; k < base_dim; ++k)
      buffer[k] = get_matrix(exponents, i, k);
    /* MPI_Send */
    #pragma omp critical 
    {
      MPI_Send(buffer, base_dim, MPI_UNSIGNED, 
               0, ROW_TAG, MPI_COMM_WORLD);
      n_bytes = (mpz_sizeinbase(As[i], 2) + 7) / 8;
      *buffer_as = 0;
      mpz_export(buffer_as, NULL, 1, 1, 1, 0, As[i]);
      MPI_Send(buffer_as, n_bytes, MPI_UNSIGNED_CHAR, 
               0, AS_TAG, MPI_COMM_WORLD);
      /* Sezione di controllo segnale di stop */
      /* ... */
    }
  }
\end{lstlisting}
L'invio del vettore degli esponenti e del vettore degli (A + s)
avviene solo al termine di ogni blocco. Come nella versione seriale, 
per minimizzare la dimensione delle strutture dati, l'algoritmo
procede per blocchi di dati di dimensione \texttt{block\char`_size}. Siccome
le primitive di MPI non sono thread safe, è stato necessario eseguire
la parte di invio dei dati in mutua esclusione tra
thread (omp critical). Come detto nel paragrafo sul master, si nota
che sono presenti due tag MPI per discriminare quale
tipo di dato è stato inviato, se il vettore degli esponenti oppure il
vettore (A + s). Riportiamo sotto anche la parte di ricezione del
segnale di stop inviato dal master. Tale messaggio mpi viene ricevuto
tramite una \texttt{MPI\char`_Irecv} asincrona. Verifichiamo la
ricezione del segnale attraverso la primitiva \texttt{MPI\char`_Test}
ogni volta che inviamo i risultati di un blocco di dati. Riportiamo di
seguito le porzioni di codice a riguardo:
\begin{lstlisting}
/* ... */
MPI_Request request;
MPI_Irecv(&stop_signal, 1, MPI_CHAR, 0, 0, MPI_COMM_WORLD, &request);
/* ... */

/* ... Al termine di un blocco */
#pragma omp critical 
{
  /* Sezione di invio ... */
  if(stop_flag == 0)
    MPI_Test(&request, &stop_flag, &status); 
}
/* ... */
\end{lstlisting}
