
%************************************************
\section{Algoritmo parallelizzato}
\label{sec:parallelo}
%************************************************

L'algoritmo ha una particolarit\'{a} molto utile sotto l'aspetto della complessita temporale:
la sezione del codice deputata alla valutazione del polinomio $Q(A)$ è fortemente parallelizzabile.
Per ottenere le congruenze necessarie dobbiamo scomporre un numero notevole di polinomi valutati 
sulla base di fattori FB.
Questa parte del lavoro pu\'{o} essere fortemente parallelizzata in quanto aumentando 
il lavoro demandato ai singoli processori è essenzialmente quello di valutare e scomporre i polinomi
su intervalli diversi; questo permette di demandare un notevole calcolo ai nodi paralleli e quindi uno speed-up notevole all'aumentare del numero di nodi.
per quanto la parte appena successiva, quella di eliminazione gaussiana è intrinsecamente seriale.


%\lstinputlisting{smart_sieve.c}
\lstinputlisting[language=c++]{../../parallel/src/smart_sieve.c}

