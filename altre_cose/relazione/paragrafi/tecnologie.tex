%************************************************
\section{Tecnologie utilizzate}
\label{sec:tecnologie}
%************************************************

\begin{fushleft}

\subsection{OpenMPI}
\label{subsec:openmpi}
OpenMPI è un'implementazione open source di MPI (\ref{subsec:mpi}).
È in grado di gestire la comunicazione fra processi con una moltitudine di tecnologie, fra cui TCP per la comunicazione fra processi in esecuzione su macchine facenti parte di una rete di tipo classico, e memoria condivisa per la comunicazione (molto più rapida) fra processi in esecuzione sulla stessa macchina.
L'esecuzione su macchine multiple è gestita tramite il protocollo SSH: la macchina su cui viene lanciato il processo contatterà le altre tramite protocollo SSH, e lancerà opportunamente il programma richiesto.
La libreria funziona assumendo che le macchine abbiano un utente con lo stesso nome e con lo stesso contenuto della home (tenterà infatti di contattare tramite SSH la macchina con lo stesso nome utente da cui è stato lanciato il processo, e cercherà l'eseguibile nello stesso path della macchina di orgine).
Al fine di evitare possibili problemi in questo senso è opportuno utilizzare uno dei tanti sistemi in grado di sincronizzare il contenuto dei dischi, o se possibile montare la home su un file system di rete.

%Aggiungere qui la spiegazione di comunicatori e altre cose del genere
Di seguito verranno esposte le primitive più importanti della libreria.
\begin{itemize}
\item MPI\_Init: Inizializza la libreria, va chiamata prima di utilizzare qualunque altra funzione di MPI.
\item MPI\_Comm\_size: Ritorna il numero di processi all'interno del comunicatore specificato.
\item MPI\_Comm\_rank: Ritorna il rank del processo all'interno del comunicatore specificato.
\item MPI\_Abort: Termina tutti i processi nel comunicatore specificato.
\item MPI\_Get\_processor: Ritorna il nome del processore che sta eseguendo il processo.
\item MPI\_Finalize: Finalizza la libreria, dopo questa chiamata non è più possibile utilizzare funzioni di libreria.
\item MPI\_Send: Send bloccante classica, la funzione ritorna quando il buffer contenente i dati da spedire è riutilizzabile.
\item MPI\_Recv: Receive bloccante classica, la funzione ritorna quando il buffer contiene i dati ricevuti ed è utilizzabile.
\item MPI\_Isend: Send non bloccante, ritorna un id della richiesta, utilizzabile per verificare lo stato dell'operazione.
\item MPI\_Irecv: Receive non bloccante, ritorna un id della richiesta, utilizzabile per verificare lo stato dell'operazione.
\item MPI\_Test: Controlla se l'operazione richiesta è terminata o meno.
\item MPI\_Wait: Attende il termine dell'operazione richiesta.
\item MPI\_Pack e MPI\_Unpack: Impacchettano e spacchettano dati da spedire.
\end{itemize}
%Continuare

\subsection{OPENMP}
\label{subsec:openmp}
A differernza di OPENMPI, OPENMP è una libreria per lo sviluppo di applicazioni parallele 
in sistemi a memoria condivisa, nel nostro caso per processi multithread che lavoreranno su cpu
multicore.
Per specificare quali parti del codice debbano essere splittate sui core senza race condition, viene usata la direttiva \#pragma omp parallel, per indicare una sezione critica invece viene utilizzata la direttiva \#pragma omp critical

Le funzioni principali sono elencate di seguito:
\begin{itemize}
\item omp\_get\_num\_procs: Restituisce il numero di processori disponibili quando viene chiamata la funzione. 
\item omp\_get\_num\_threads: Restituisce il numero di thread nell'area parallela.
\item omp\_get\_wtime: Restituisce un valore in secondi del tempo trascorso da un certo punto. 
\end{itemize}

\subsection{GMP}
\label{subsec:gmp}
GMP \'{e} una libreria libera per l'utilizzo di aritmetica a precizione arbitraria e fa parte
del progetto GNU $($ Gnu Multile Precision $)$.
L'unica limitazione alla dimensione dei valori assumibili da una variabile mpz è la dimensione della
memoria del dispositivo.
GMP viene utilizzata per algoritmi crittografici, applicazioni relative alla sicurezza delle reti
e per sistemi di algebra numerica.
Internamente GMP rappresenta gli interi come dei vettori.
L'utilizzo di questa libreria all'interno dell'algoritmo è stato necessario sia per la rappresentazione
dell'intero da fattorizzare e per i suoi fattori primi, se pensiamo che RSA utilizza numeri con più
di 100 cifre viene normale utilizzare rappresentazioni in precisione arbitraria, che per gli esponenti
all'interno delle fattorizzazioni: per quanto potesse sembrare poco probabile si è notato che gli esponenti 
nelle fattorizzazioni raggiungono dimensioni elevate, si è quindi dimostrato necessario utilizzare variabili
mpz per rappresentare anche questi.

Di seguito verranno esposte le primitive più importanti della libreria.
\begin{itemize}
\item gmp\_init: Questa funzione va chiamata ad ogni dichiarazione per inizializzare la variabile
\item gmp\_add: Somma due mpz
\item gmp\_sub: Sottrae due mpz
\item gmp\_mul: Moltiplica due mpz
\item gmp\_div: Divide due mpz
\item gmp\_mod: Calcola il resto della divisione intera tra due mpz
\item gmp\_legendre: Calcola il valore del simbolo di Legendre
\item gmp\_sqrt: Radice quadrata
\item gmp\_pow: Potenza
\item gmp\_gcd: Massimo Comun Divisore
\item gmp\_func\_ui: specificando \_ui indichiamo che il secondo parametro è un intero senza segno
\end{itemize}




\end{fushleft}