3%************************************************
\section{Sviluppi futuri}
\label{sec:sviluppi}
%************************************************


L'algoritmo si presta a diversi sviluppi orientati sia al miglioramento delle prestazioni che al miglioramento dell'interfaccia
con l'utente.

In merito alle prestazioni la prima modifica utile che potrebbe essere apportata è la seguente:
l'architettura del sistema in questo momento è di tipo master slave, nella fattispecie il master svolge semplicemente il compito di suddividere  
i dati e di recuperarli per poi eseguire la parte seriale dell'algoritmo, ossia l'eliminazione Gaussiana.
Sarebbe utile modificare l'algoritmo affinchè nel tempo di attesa anche il processo master eseguisse la parte degli slave così da ottimizzare
l'uso delle cpu.

Considerando l'esecuzione dell'algoritmo in parallelo sui nodi di un supercomputer occorre considerare il problema del walltime, ossia
del tempo massimo assegnato ad un processo dal gestore di code.
Se il processo supera il walltime senza aver terminato correttamente l'esecuzione viene ucciso.
Questo è un problema piuttosto notevole se si tratta di algoritmi che richiedono diversi giorni di calcolo su determinati dati; occorre quindi
permettere all'algoritmo di crearsi dei checkpoint facendo un salvataggio dei dati intermedi per poi riprendere la computazione dal punto in 
cui era stato bloccato.