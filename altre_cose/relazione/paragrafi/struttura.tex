%************************************************
\section{Struttura della repository}
\label{sec:struttura}
%************************************************
La repository del progetto è ospitata sul sito GitHub, liberamente
scaricabile all'indirizzo \url{https://github.com/UnProgrammatore/CCQ}.
Nel progetto sono proposte due versioni del crivello quadratico:
seriale e parallelo. Nella directory \texttt{/serial} è contenuto 
il codice dell'algoritmo seriale mentre in \texttt{/parallel} la versione
parallela. Nel capitolo sarà trattata dettagliatamente la versione
parallela dell'algoritmo parlando di come di compila ed esegue su un
cluster di computer.
\subsection{Suddivisione dei sorgenti}
Entrambe le versioni, seriale e parallela, hanno una struttura dei sorgenti simile.
Ogni file contiene funzioni che occorrono in una determinata fase del
crivello. La directory dei sorgenti è \texttt{/src}. Il programma è
suddiviso in 7 file:
\begin{itemize}
\item \texttt{base\char`_fattori.c}
\item \texttt{eratostene.c}
\item \texttt{linear\char`_algebra.c}
\item \texttt{quadratic\char`_sieve.c}
\item \texttt{trivial\char`_fact.c}
\item \texttt{vector.c}
\item \texttt{matrix.c}
\end{itemize}
Il file \texttt{base\char`_fattori.c} contiene le funzioni per il
calcolo della base di fattori. Il file \texttt{eratostene.c} contiene
la funzione per il calcolo dei primi $n$ numeri primi mediante il
crivello di Eratostene. In \texttt{linear\char`_algebra.c} sono
contenute  le funzioni per l'eliminazione gaussiana, eseguita sulla
matrice degli esponenti. Nello stesso file sono presenti anche le
funzioni che permettono il calcolo delle congruenze nella parte
finale dell'algoritmo. \texttt{quadratic\char`_sieve.c} è il file
principale del programma. Da questo vengono richiamate 
le funzione che eseguono le varie fasi
dell'algoritmo. \texttt{trivial\char`_fact.c} è il file con le
funzioni che effettuano la fattorizzazione per tentativi necessarie
nella parte iniziale dell'algoritmo. \texttt{vector.c} e
\texttt{matrix.c} contengono funzioni di servizio per la
gestione delle strutture dati utilizzati. \texttt{vector.c} gestisce
la struttura dati vettore dinamico istanziata per vari tipi di dato:
\texttt{unsigned int}, \texttt{unsigned long} e \texttt{mpz\char`_t}. 
\texttt{matrix.c} gestisce matrici di dimensione dinamica in entrambe le
dimensioni. Come per i vettori è presente la variante \texttt{unsigned
int}, \texttt{unsigned long} e \texttt{mpz\char`_t}.
Sono presenti header file per ognuno dei sorgenti c nella directory 
\texttt{/include}.
\subsection{Compilazione}
Per la compilazione del codice sono presenti tre makefile.
Il primo, chiamato semplicemente ``Makefile'', occorre per compilare
generalmente il codice per un pc linux. Un avvertenza: è necessario
creare le directory \texttt{/lib} ed \texttt{/executables} nella cartella principale
dell'algoritmo (es: \texttt{/parallel/executables}). Queste cartelle conterranno i
prodotti della compilazione, executables per gli eseguibili e lib per
gli object file.
Gli ulteriori makefile, ``Makefile.intel'' e ``Makefile.gnu'', sono stati
utilizzati per la compilazione del codice sul Galileo utilizzando
il compilatore mpi di intel e gnu.
\subsection{Esecuzione del programma parallelo}
Il main del programma (che è da considerare puramente
a scopo di debugging e non main di un applicazione definitiva)
ha la seguente interfaccia a linea di comando:
\begin{lstlisting}[language=bash]
  $ mpirun [opzioni_mpi] executables/qs [n1] [n2] [crivello] [intervallo] [blocco]
\end{lstlisting}
\begin{itemize}
\item[\texttt{n1}, \texttt{n2}] A scopo di debugging l'applicazione prende
in input due numeri che moltiplica tra loro producendo il numero $N$
da fattorizzare. Porre uno dei due numeri a 1 per inserire
direttamente l'$N$ da fattorizzare;
\item[\texttt{crivello}] Per calcolare la
base di fattori si calcolano prima i numeri primi da  
0 a \texttt{crivello} mediante il crivello di eratostene (nota: la
base di fattori sarà molto più piccola di questo parametro);
\item[\texttt{intervallo}] Dimensione dell'insieme insieme degli $A$
per i queali ogni salve fattorizza i rispettivi $Q(A)$; 
\item[\texttt{blocco}] Dato un intervallo ad uno slave questo lo separa in
altri sottointervalli che assegna ad ogni thread. Ogni thread calcola
alla volta \texttt{blocco} valori di $A$;
\item[\texttt{fact\char`_print}] Parametro che indica ogni quante fattorizzazioni trovate, il programma produce un
output su stdout per mostrare la percentuale di completamento.
\end{itemize}
\begin{lstlisting}[language=bash]
$ mpirun -np 4 executables/qs-intel 1 18567078082619935259 4000 10000000 10 1000 50
\end{lstlisting}
\subsection{Galileo}
L'esecuzione dell'algortimo è avvenuto principalmente su due diversi
sistemi: il cluster del dipartimento di fisica dell'Università di
Parma e il supercumputer Galileo presente al Cineca
(http://www.hpc.cineca.it/hardware/galileo). Riportiamo di seguito le
specifiche tecniche del sistema prese direttamente dal sito web del
cineca:
\\
\\
\begin{tabular}{l l}
Modello: & IBM NeXtScale \\
Architettura: & Linux Infiniband Cluster \\
Nodi: & 516 \\
Processori: & 2 8-cores Intel Haswell 2.40 GHz per node \\
Cores: & 16 cores/node, 8256 cores in total \\
Acceleratori grafici: & 2 Intel Phi 7120p per node on 384 nodes (768
in total); \\
                      & 2 NVIDIA K80 per node on 40 nodes \\
RAM: & 128 GB/node, 8 GB/core \\
Internal Network: & Infiniband with 4x QDR switches \\
Disk Space: & 2.000 TB of local scratch \\
Peak Performance: & 1.000 TFlop/s (da definire) \\
Sistema Operativo: & CentOS 7.0 \\
\end{tabular}
\subsection{Esecuzione e gestore delle code}
Qui riportata una sezione dello script (\texttt{script.sh}) per la sottomissione dei job sul Galileo.
La sottomissione avviene mediante il comando \texttt{qsub script.sh}. Per ulteriori informazioni
sul gestore delle code utilizzato nel Galileo si rimanda al sito del
cineca.
\begin{lstlisting}[language=bash]
#PBS -A INFNG_test 
#PBS -l select=1:ncpus=16:mem=100gb:mpiprocs=2+7:ncpus=16:mem=100gb:mpiprocs=1 
#PBS -l walltime=24:00:00 
#PBS -N CCQi-n8-65l
name="CCQi-n8-65l"

cat $PBS_NODEFILE

module load intel intelmpi
\end{lstlisting}
Queste riportate sono direttive per il gestore delle code. Notare il
comando \texttt{module load intel intelmpi} che consente di utilizzare il 
compilatore mpi di intel. 
\begin{lstlisting}[language=bash]
#PBS -l select=1:ncpus=16:mem=100gb:mpiprocs=2+7:ncpus=16:mem=100gb:mpiprocs=1 
\end{lstlisting}
Vediamo nel dettaglio la direttiva per l'ottenimento delle risorse
richieste. Ricordiamo che sul nodo master vogliamo l'esecuzione di due processi
MPI. Per fare questo richiediamo un nodo solo (\texttt{select=1:ncpus=16:mem=100gb:mpiprocs=2)} con 16 core, sul quale richiediamo
due processi MPI. Il simbolo ``+'' nella stringa serve per specificare ulteriori risorse da richiedere infatti
con \texttt{7:ncpus=16:mem=100gb:mpiprocs=1} stiamo richiedendo 7
nodi, 16 core ognuno e un solo processo MPI per nodo.
