%************************************************
\section{Struttura del programma}
\label{sec:struttura}
%************************************************
\subsection{Suddivisione dei sorgenti}
L'implementazione del crivello quadratico proposto si suddivide in due
versioni: seriale e parallelo.
Entrambe le versioni hanno una struttura dei sorgenti simile.
Ogni file contiene funzioni che occorrono in una determinata fase del
crivello. La directory dei sorgenti è "/src". Il programma è
suddiviso in 7 file:

\begin{itemize}
\item \texttt{base\char`_fattori.c}
\item \texttt{eratostene.c}
\item \texttt{linear\char`_algebra.c}
\item \texttt{quadratic\char`_sieve.c}
\item \texttt{trivial\char`_fact.c}
\item \texttt{vector.c}
\item \texttt{matrix.c}
\end{itemize}

Il file \texttt{base\char`_fattori.c} contiene le funzioni per il
calcolo della base di fattori. Il file \texttt{eratostene.c} contiene
la funzione per il calcolo dei primi $n$ numeri primi mediante il
crivello di Eratostene. In \texttt{linear\char`_algebra.c} sono
contenute  le funzioni per l'eliminazione gaussiana, eseguita sulla
matrice degli esponenti. Nello stesso file sono presenti anche le
funzioni che permettono il calcolo delle congruenze nella parte
finale dell'algoritmo. \texttt{quadratic\char`_sieve.c} è il file
principale del programma. Da questo vengono richiamate 
le funzione che eseguono le varie fasi
dell'algoritmo. \texttt{trivial\char`_fact.c} è il file con le
funzioni che effettuano la fattorizzazione per tentativi necessarie
nella parte iniziale dell'algoritmo. \texttt{vector.c} e
\texttt{matrix.c} contengono funzioni di servizio per la
gestione delle strutture dati utilizzati. \texttt{vector.c} gestisce
la struttura dati vettore dinamico istanziata per vari tipi di dato:
\texttt{unsigned int}, \texttt{unsigned long} e \texttt{mpz\char`_t}. 
\texttt{matrix.c} gestisce matrici di dimensione dinamica in entrambe le
dimensioni. Come per i vettori è presente la variante \texttt{unsigned
int}, \texttt{unsigned long} e \texttt{mpz\char`_t}.
Sono presenti header file per ognuno dei sorgenti c nella directory 
"/include".
\subsection{Compilazione}
Per la compilazione del codice sono presenti tre makefile.
Il primo, chiamato semplicemente "Makefile", occorre per compilare
generalmente il codice per un pc linux. Un avvertenza: è necessario
creare le directory "lib" ed "executables" nella cartella principale
dell'algoritmo (es: parallel/executables). Queste cartelle conterranno i
prodotti della compilazione, executables per gli eseguibili e lib per
gli object file.
Gli ulteriori makefile, Makefile.intel e Makefile.gnu, sono stati
utilizzati per la compilazione del codice sul Galileo utilizzando
il compilatore mpi di intel e gnu.
\subsection{Esecuzione del programma parallelo}
Il main del programma (che è da considerare puramente
a scopo di debugging e non main di un applicazione definitiva)
ha la seguente interfaccia a linea di comando:
\begin{lstlisting}[language=bash]
  $ mpirun [opzioni_mpi] executables/qs [n1] [n2] [dim_crivello_eratostene] [dim_intervallo] [dim_blocco]
\end{lstlisting}
\begin{itemize}
\item[\texttt{n1}, \texttt{n2}] A scopo di debugging l'applicazione prende
in input due numeri che moltiplica tra loro producendo il numero $N$
da fattorizzare. Porre uno dei due numeri a 1 per inserire
direttamente l'$N$ da fattorizzare;
\item[\texttt{dim\char`_crivello\char`_eratostene}] Per calcolare la
base di fattori si calcolano prima i numeri primi da  
0 a \texttt{dim\char`_crivello\char`_eratostene} (nota: la base di fattori sarà molto più
piccola di questo parametro);
\item[\texttt{dim\char`_intervallo}] dimensione insieme di A per i queali ogni salve
fattorizza i rispettivi $Q(A)$; 
\item[\texttt{dim\char`_blocco}] Dato un intervallo ad uno slave questo lo separa in
altri sottointervalli che assegna ad ogni thread. Ogni thread calcola
alla volta \texttt{dim\char`_blocco} valori di $A$;
\item[\texttt{fact\char`_print}] Parametro che indica ogni quante fattorizzazioni trovate, il programma produce un
output su stdout per mostrare la percentuale di completamento.
\end{itemize}
\begin{lstlisting}[language=bash]
$ mpirun executables/qs-intel 1 18567078082619935259 4000 10000000 10 1000 50
\end{lstlisting}
