%************************************************
\section{Crivello Quadratico}
\label{sec:crivello}
%************************************************

\begin{flushleft}


Il Crivello quadratico è assieme al Crivello coi Campi di Numeri l'algoritmo di fattorizzazione
pi\'{u} veloce ad oggi conosciuto.
Il costo computazionale risulta essere asintoticamente subesponenziale nell'ordine di
$O(exp(sqrt(log(N))log(log(N))))$.
Per comprendere l'algoritmo occorre presentare almeno in parte la tecnica di fattorizzazione di
Fermat.
Fermat osserv\'{o} che per trovare una fattorizzazione di un numero N si possono trovare due 
numeri X e Y tali per cui $X^2-Y^2 = N$ trovando quindi $(X+Y)(X-Y)$.
Qualora $X+Y$ o $X-Y$ non risultassero fattori banali, ossia 1 o N stesso, avremmo trovato una
fattorizzazione completa per N.

Il Crivello Quadratico parte da questa semplice idea sviluppandola per ottenere un algoritmo 
alquanto efficiente e piuttosto articolato.

\subsection*{algoritmo}
Dato un numero N intero, il crivello, a differenza dell'algoritmo di Fermat, cerca dei valori 
X e Y tali per cui valga la relazione $X \equiv Y \mod N$, successivamente ricerca il massimo comun
divisore tra $(X-Y,N)$.
Iniziamo calcolando una base di fattori primi FB di dimensione $k=k(N)$.
Questo calcolo avviene mediante la scrematura di una base data dal crivello di eratostene.
La discriminante per i fattori primi \emph{p} della base è che essi abbiano simbolo di Legendre 
$(N|P)=1$, ossia siano tali per cui N sia un residuo quadratico modulo p
Sia ora $s=\sqrt[2]{N}$ impostiamo il seguente polinomio: $Q(A)=(A+s)^2-N$.
Siamo certi che $Q(A)\equiv N$ sia un quadrato perfetto, il lavoro ora consiste nel 
trovare dei valori di A tali per qui $Q(A)$ si fattorizzi completamente sulla base di fattori
precedentemente calcolata.
Quando uno di questi polinomi si fattorizza completamente sulla base FB creiamo un vettore
$v=(\alpha_1,\alpha_2, .. , \alpha_n)$ dove ogni $\alpha_i$ rappresenta l'esponente dell'i-esimo numero primo nella fattorizzazione di $Q(A)$.
Calcoliamo quindi un altro vettore $v_2=(\alpha_1,\alpha_2, .. , \alpha_n)_2$ ossia il vettore 
degli esponenti in base binaria.
Nel caso banale in cui $v_i$ fosse identicamente nullo, allora ogni primo avrebbe un esponente
pari, in questo caso sarebbe un quadrato perfetto e, quindi, avremmo trovato una congruenza del tipo $X^2 \equiv Y^2\mod N$.
Anche se questo accadesse potremmo aver ottenuto una fattorizzazione banale, si prosegue quindi
con la parte dell'algoritmo che riguarda prettamente l'algebra lineare.
I vettori $v_2i$ vengono inseriti tutti in una grossa matrice.
Dall'algebra sappiamo che, data una matrice di k colonne necessita di almeno k+1 righe per 
ottenere almeno una dipendenza lineare.
Nel nostro caso occorrerà quindi trovare almeno $k+m$ con $m\geq1$ per poter ottenere un numero di righe tale da permettere di trovare almeno una dipendenza lineare.

Consideriamo le seguenti matrici:

\begin{center}

\begin{tabular}{ |l|c|c|c|c| }
\hline
$v_1$ & 3 & 4 & 2 & 7\\
\hline
$v_2$ & 3 & 2 & 2 & 2\\
\hline
$v_3$ & 5 & 2 & 3 & 1\\
\hline
$v_4$ & 6 & 1 & 3 & 3\\
\hline
$v_5$ & 2 & 2 & 2 & 3\\
\hline
\end{tabular}

\vspace{1cm} 

\begin{tabular}{|l|c|c|c|c|}
\hline
$v_1^2$ & 1 & 0 & 0 & 1\\
\hline
$v_2^2$ & 1 & 0 & 0 & 0\\
\hline
$v_3^2$ & 1 & 0 & 1 & 1\\
\hline
$v_4^2$ & 0 & 1 & 1 & 1\\
\hline
$v_5^2$ & 0 & 0 & 0 & 1\\
\hline
\end{tabular}

\end{center}

Notiamo che dalla combinazione lineare di $v1_2$ $v2_2$ e $v4_2$ otteniamo un vettore nullo.
Andiamo quindi a considerare ora i relativi vettori non modulati $ v1 v2 e v5$ e sommiamoli tra
loro, ottenendo $v_125=(8,8,6,12)$.
Se ora consideriamo la congruenza 
\begin{center}
   $Q(A_1)Q(A_2)Q(A_5) \equiv 2^8*3^8*5^6*7^12 \mod N$
\end{center}
Considerando il membro di sinistra come la X e quello di destra come la Y della nostra relazione iniziale, possiamo ricercare la congruenza desiderata e tentare di fattorizzare N.












\end{flushleft}


\lipsum[1]




