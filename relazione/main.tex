\documentclass[10pt,%                       % corpo del font principale
               a4paper,%                    % carta A4
               oneside,%                    % solo fronte
%              twoside,%                    % fronte-retro
	       titlepage,%                  % pagina del titolo
               ]{article}                   % classe report di KOMA-Script;           
\usepackage[T1]{fontenc}                    % codifica dei font:
                                            % NOTA BENE! richiede una distribuzione *completa* di LaTeX,
                                            % per esempio TeXLive o MiKTeX *complete*
\usepackage[utf8]{inputenc}                 % codifica di input; anche [latin1] va bene
                                            % NOTA BENE! va accordata con le preferenze dell'editor
\usepackage[english,italian]{babel}         % per scrivere in italiano e in inglese;
                                            % l'ultima lingua (l'italiano) risulta predefinita
\usepackage[binding=5mm]{layaureo}          % margini ottimizzati per l'A4; rilegatura di 5 mm
\usepackage{indentfirst}                    % rientra il primo capoverso di ogni sezione
\usepackage{booktabs}                       % tabelle
\usepackage{tabularx}                       % tabelle di larghezza prefissata
\usepackage{graphicx}                       % immagini
\usepackage{subfig}                         % sottofigure, sottotabelle
\usepackage{caption}                        % didascalie
\usepackage{listings}                       % codici
\usepackage[font=small]{quoting}            % citazioni
\usepackage{amsmath,amssymb,amsthm}         % matematica
\usepackage[italian]{varioref}              % riferimenti completi della pagina
\usepackage{mparhack,fixltx2e,relsize}      % finezze tipografiche
\usepackage[style=philosophy-modern,hyperref,backref,square,natbib]{biblatex}
                                            % eccellente pacchetto per la bibliografia;
                                            % produce uno stile di citazione autore-anno; 
                                            % lo stile "numeric-comp" produce riferimenti numerici
\bibliography{bibliografia}                 % database di biblatex
\usepackage[dvipsnames]{xcolor}             % colori
\usepackage{lipsum}                         % testo fittizio
\usepackage{eurosym}                        % simbolo dell'euro
\usepackage{hyperref}                       % collegamenti ipertestuali
\usepackage{bookmark}                       % segnalibri
\usepackage{frontespizio}		    % frontespizio

%*********************************************************************************
% impostazioni-articolo.tex
% file che contiene le impostazioni dell'articolo
%*********************************************************************************


%*********************************************************************************
% Comandi personali
%*********************************************************************************
\newcommand{\myName}{Andrea Chiastra, Andrea Segalini, Lorenzo Pattarini} % autore
\newcommand{\myTitle}{Configurazione di un cluster per 
l'algoritmo del crivello quadratico}    				  % titolo
\date{}                                                             % nessuna data

\title{\myTitle}
\author{\myName}

\newcommand{\mex}{\mathrm{mex}}



%*********************************************************************************
% Impostazioni di amsmath, amssymb, amsthm
%*********************************************************************************

% comandi per gli insiemi numerici (serve il pacchetto amssymb)
\newcommand{\numberset}{\mathbb} 
\newcommand{\N}{\numberset{N}} 
\newcommand{\R}{\numberset{R}} 

% un ambiente per i sistemi
\newenvironment{sistema}%
  {\left\lbrace\begin{array}{@{}l@{}}}%
  {\end{array}\right.}

% definizioni (serve il pacchetto amsthm)
\theoremstyle{definition} 
\newtheorem{definizione}{Definizione}

% teoremi, leggi e decreti (serve il pacchetto amsthm)
\theoremstyle{plain} 
\newtheorem{teorema}{Teorema}
\newtheorem{legge}{Legge}
\newtheorem{decreto}[legge]{Decreto}
\newtheorem{murphy}{Murphy}[section]



%*********************************************************************************
% Impostazioni di biblatex
%*********************************************************************************
\defbibheading{bibliography}{%
\phantomsection 
\addcontentsline{toc}{section}{\refname}
\section*{\bibname\markboth{\MakeUppercase{\refname}}{\MakeUppercase{\refname}}}}



%*********************************************************************************
% Impostazioni di listings
%*********************************************************************************
\lstset{language=C++,
    keywordstyle=\color{RoyalBlue},%\bfseries,
    basicstyle=\small\ttfamily,
    %identifierstyle=\color{NavyBlue},
    commentstyle=\color{Green}\ttfamily,
    stringstyle=\rmfamily,
    numbers=none,%left,%
    numberstyle=\scriptsize,%\tiny
    stepnumber=5,
    numbersep=8pt,
    showstringspaces=false,
    breaklines=true,
    frameround=ftff,
    frame=single
} 



%*********************************************************************************
% Impostazioni di hyperref
%*********************************************************************************
\hypersetup{%
    hyperfootnotes=false,pdfpagelabels,
    %draft,	% = elimina tutti i link (utile per stampe in bianco e nero)
    colorlinks=true, linktocpage=true, pdfstartpage=1, pdfstartview=FitV,%
    % decommenta la riga seguente per avere link in nero (per esempio per la stampa in bianco e nero)
    %colorlinks=false, linktocpage=false, pdfborder={0 0 0}, pdfstartpage=1, pdfstartview=FitV,% 
    breaklinks=true, pdfpagemode=UseNone, pageanchor=true, pdfpagemode=UseOutlines,%
    plainpages=false, bookmarksnumbered, bookmarksopen=true, bookmarksopenlevel=1,%
    hypertexnames=true, pdfhighlight=/O,%nesting=true,%frenchlinks,%
    urlcolor=webbrown, linkcolor=RoyalBlue, citecolor=webgreen, %pagecolor=RoyalBlue,%
    %urlcolor=Black, linkcolor=Black, citecolor=Black, %pagecolor=Black,%
    pdftitle={\myTitle},%
    pdfauthor={\textcopyright\ \myName},%
    pdfsubject={},%
    pdfkeywords={},%
    pdfcreator={pdfLaTeX},%
    pdfproducer={LaTeX with hyperref and ClassicThesis}%
}



%*********************************************************************************
% Impostazioni di graphicx
%*********************************************************************************
\graphicspath{{immagini/}} % cartella dove sono riposte le immagini



%*********************************************************************************
% Impostazioni di xcolor
%*********************************************************************************
\definecolor{webgreen}{rgb}{0,.5,0}
\definecolor{webbrown}{rgb}{.6,0,0}



%*********************************************************************************
% Impostazioni di caption
%*********************************************************************************
\captionsetup{tableposition=top,figureposition=bottom,font=small,format=hang,labelfont=bf}



%*********************************************************************************
% Altro
%*********************************************************************************

% [...] ;-)
\newcommand{\omissis}{[\dots\negthinspace]}

% eccezioni all'algoritmo di sillabazione
\hyphenation{Fortran ma-cro-istru-zio-ne nitro-idrossil-amminico}
               	    % file con le impostazioni personali

\begin{document}
\pagestyle{headings} 
%******************************************************************
% Materiale iniziale
%******************************************************************
%*******************************************************
% Frontespizio
%*******************************************************
\begin{frontespizio}
\Universita{Parma}
\Dipartimento{Matematica e informatica}
\Logo[1.5cm]{logo}
\Corso[Corso di laurea]{Informatica}
\Annoaccademico{2014--2015}
\Titoletto{Progetto per il corso di reti di calcolatori}
\Titolo{Crivello Quadratico \\ Parallelo}
\NCandidati{Studenti}
\Candidato[]{Lorenzo Pattarini}
\Candidato[]{Andrea Chiastra}
\Candidato[]{Andrea Segalini}
\end{frontespizio}

%*******************************************************
% Indici
%*******************************************************
\pdfbookmark{\contentsname}{tableofcontents}
\setcounter{tocdepth}{2}
\tableofcontents
\markboth{\contentsname}{\contentsname} 

%*******************************************************
% Elenco delle figure
%*******************************************************    
%\phantomsection
%\pdfbookmark{\listfigurename}{lof}
%\listoffigures

%*******************************************************
% Elenco delle tabelle
%*******************************************************
%\phantomsection
%\pdfbookmark{\listtablename}{lot}
%\listoftables
        

%% !TEX encoding = UTF-8
% !TEX TS-program = pdflatex
% !TEX root = ../Articolo.tex
% !TEX spellcheck = it-IT

%*******************************************************
% Sommario+Abstract
%*******************************************************
\phantomsection
\pdfbookmark{Sommario}{Sommario}
\section*{Sommario}

\lipsum[1]

\selectlanguage{english}
\pdfbookmark{Abstract}{Abstract}
\section*{Abstract}

\lipsum[2]

\selectlanguage{italian}


%******************************************************************
% Materiale principale
%******************************************************************
%************************************************
\section{Introduzione e obiettivi}
\label{sec:introduzione}
%************************************************

-sicurezza
-perchè rsa
-perchè problema della fattorizzazione
-perchè quest'algoritmo
-Obiettivo implementazione


-fattorizzazione
-conseguenze del problema (rsa)
-sicurezza
-Rompere rsa ( implementazione parallela qs rsa-challenge)
-ottimizzazione dell'algoritmo
		( cluster // uso linux-mpi )


Per il teorema fondamentale dell'Arimetica dato un numero N non primo, il quale possiede quindi dei divisori non banali, esiste ed è unica, prescindendo dall'ordine dei fattori, la sua fattorizzazione esprimibile come prodotto di numeri primi elevati ad opportune potenze.
Il problema della fattorizzazione di numeri interi viene affrontato sin dalle elementari per trovare relazioni tra due numeri quali il massimo comun divisore o il minimo comune multiplo, ognuno di noi ha quindi presente di cosa si tratti, per lo meno ad un livello intuitivo.
Quando ci si addentra nell'ostico compito di fattorizzare numeri che contengono un numero di cifre nell'ordine delle centinaia il problema, però, si dimostra essere molto più difficile di quanto si potesse pensare analizzandolo in maniera intuitiva.
Sotto l'aspetto della teoria della complessità computazionale il problema risulta essere esponenziale, subesponenziale nel caso di alcuni algoritmi particolari.
Sulla difficoltà di questo problema si basa un famosissimo algoritmo di cifratura: RSA.

RSA fa parte di quella branca della crittografia moderna che prende il nome di crittografia asimmetrica.
A differenza della crittografia classica o simmetrica, quella asimmetrica prevede l'esistenza di due tipi di chiavi, una pubblica ed una privata.
Questo approccio permette situazioni del seguente tipo:
\begin{itemize}


\item Alice vuole spedire un messaggio a Bob di modo che solo Bob possa leggerlo, userà quindi la chiave pubblica di Bob per cifrare il messaggio sapendo che solo Bob con la sua chiave privata potrà decifrarlo.

\item Alice vuole spedire un messaggio a Bob di modo che, non solo Bob sia l'unico a poterlo leggere, ma che esso abbia la certezza che il mittente del messaggio sia proprio Alice.
Alice quindi cifra il messaggio con la propria chiave privata e successivamente con quella pubblica di Bob. In questo modo all'atto della ricezione, Bob potrà applicare la propria chiave privata e la chiave pubblica di Alice per poter decifrare il messaggio essendo certo della provenienza dello stesso.

\end{itemize}

Il funzionamento di RSA non è particolarmente complesso.
Supponiamo che Alice e Bob stiano avendo un dialogo segreto, ossia non vogliono che un eventuale haker possa intercettare la loro comunicazione e comprenderne il significato.
Sia quindi \emph{M} il messaggio che si vogliono scambiare.
Ognuno di loro sceglie due numeri primi \emph{p} e \emph{q} li moltiplica tra di loro ottenendo \emph{N=p.q}.
In seguito calcolano $\varphi( N)  = ( p-1)(q-1)$.
Scelgono infine un numero \emph{e} coprimo con $\varphi(N)$ e minore dello stesso e calcolano \emph{d} tale per cui $e.d\equiv 1 mod(N)$.
Ora $(N, e)$ è la chiave pubblica, mentre quella privata è $(N, d)$.
Il messaggio visibile sulla rete è il seguente: $\exp(M,1)=\exp(M,e)mod(N)$ che verr\'{a} poi decifrato applicando una semplice esponenziazione di esponente \emph{d}, elemento della chiave pubblica del mittente.

La sicurezza di quest'algoritmo risiede nella difficoltà computazionale di fattorizzare il numero \emph{N} nei suoi fattori primi e quindi nel trovare la funzione $\varphi(N)$
che permetterebbe di trovare l'inverso moltiplicativo di \emph{e} e quindi rompere i sistema.

Nel 1991 la RSA Laboratories propose come sfida la fattorizzazione di 54 semiprimi ( prodotti di due primi ) con un numero di cifre compreso tra 100 e 617.
Ad oggi solo i 12 pi\'{u} piccoli sono stati fattorizzati e, nonostante il 2007 vide la chiusura dell'RSA Challenge in molti ancora si dilettano nel tentativo di fattorizzarli.
%************************************************
\section{Crivello Quadratico}
\label{sec:crivello}
%************************************************

\begin{flushleft}


Il Crivello quadratico è assieme al Crivello coi Campi di Numeri l'algoritmo di fattorizzazione
pi\'{u} veloce ad oggi conosciuto.
Il costo computazionale risulta essere asintoticamente subesponenziale nell'ordine di
$O(exp(sqrt(log(N))log(log(N))))$.
Per comprendere l'algoritmo occorre presentare almeno in parte la tecnica di fattorizzazione di
Fermat.
Fermat osserv\'{o} che per trovare una fattorizzazione di un numero N si possono trovare due 
numeri X e Y tali per cui $X^2-Y^2 = N$ trovando quindi $(X+Y)(X-Y)$.
Qualora $X+Y$ o $X-Y$ non risultassero fattori banali, ossia 1 o N stesso, avremmo trovato una
fattorizzazione completa per N.

Il Crivello Quadratico parte da questa semplice idea sviluppandola per ottenere un algoritmo 
alquanto efficiente e piuttosto articolato.

\subsection*{Algoritmo}
Dato un numero N intero, il crivello, a differenza dell'algoritmo di Fermat, cerca dei valori 
X e Y tali per cui valga la relazione $X \equiv Y \mod N$, successivamente ricerca il massimo comun
divisore tra $(X-Y,N)$.
Iniziamo calcolando una base di fattori primi FB di dimensione $k=k(N)$.
Questo calcolo avviene mediante la scrematura di una base data dal crivello di eratostene.
La discriminante per i fattori primi \emph{p} della base è che essi abbiano simbolo di Legendre 
$(N|P)=1$, ossia siano tali per cui N sia un residuo quadratico modulo p
Sia ora $s=\sqrt[2]{N}$ impostiamo il seguente polinomio: $Q(A)=(A+s)^2-N$.
Siamo certi che $Q(A)\equiv N$ sia un quadrato perfetto, il lavoro ora consiste nel 
trovare dei valori di A tali per qui $Q(A)$ si fattorizzi completamente sulla base di fattori
precedentemente calcolata.
Quando uno di questi polinomi si fattorizza completamente sulla base FB creiamo un vettore
$v=(\alpha_1,\alpha_2, .. , \alpha_n)$ dove ogni $\alpha_i$ rappresenta l'esponente dell'i-esimo numero primo nella fattorizzazione di $Q(A)$.
Calcoliamo quindi un altro vettore $v_2=(\alpha_1,\alpha_2, .. , \alpha_n)_2$ ossia il vettore 
degli esponenti in base binaria.
Nel caso banale in cui $v_i$ fosse identicamente nullo, allora ogni primo avrebbe un esponente
pari, in questo caso sarebbe un quadrato perfetto e, quindi, avremmo trovato una congruenza del tipo $X^2 \equiv Y^2\mod N$.
Anche se questo accadesse potremmo aver ottenuto una fattorizzazione banale, si prosegue quindi
con la parte dell'algoritmo che riguarda prettamente l'algebra lineare.
I vettori $v_2i$ vengono inseriti tutti in una grossa matrice.
Dall'algebra sappiamo che, data una matrice di k colonne necessita di almeno k+1 righe per 
ottenere almeno una dipendenza lineare.
Nel nostro caso occorrerà quindi trovare almeno $k+m$ con $m\geq1$ per poter ottenere un numero di righe tale da permettere di trovare almeno una dipendenza lineare.

Consideriamo le seguenti matrici:

\begin{center}

\begin{tabular}{ |l|c|c|c|c| }
\hline
$v_1$ & 3 & 4 & 2 & 7\\
\hline
$v_2$ & 3 & 2 & 2 & 2\\
\hline
$v_3$ & 5 & 2 & 3 & 1\\
\hline
$v_4$ & 6 & 1 & 3 & 3\\
\hline
$v_5$ & 2 & 2 & 2 & 3\\
\hline
\end{tabular}

\vspace{1cm} 

\begin{tabular}{|l|c|c|c|c|}
\hline
$v_1^2$ & 1 & 0 & 0 & 1\\
\hline
$v_2^2$ & 1 & 0 & 0 & 0\\
\hline
$v_3^2$ & 1 & 0 & 1 & 1\\
\hline
$v_4^2$ & 0 & 1 & 1 & 1\\
\hline
$v_5^2$ & 0 & 0 & 0 & 1\\
\hline
\end{tabular}

\end{center}

Notiamo che dalla combinazione lineare di $v1_2$ $v2_2$ e $v4_2$ otteniamo un vettore nullo.
Andiamo quindi a considerare ora i relativi vettori non modulati $ v1 v2 e v5$ e sommiamoli tra
loro, ottenendo $v_125=(8,8,6,12)$.
Se ora consideriamo la congruenza 
\begin{center}
   $Q(A_1)Q(A_2)Q(A_5) \equiv 2^8*3^8*5^6*7^12 \mod N$
\end{center}
Considerando il membro di sinistra come la X e quello di destra come la Y della nostra relazione iniziale, possiamo ricercare la congruenza desiderata e tentare di fattorizzare N.
\end{flushleft}






%************************************************
\section{Architettura del sistema di calcolo distribuito}
\label{sec:architettura}
%************************************************

\lipsum[1]





%************************************************
\section{Strategia di parallelizzazione dell'algoritmo}
\label{sec:strategia}
%************************************************

\lipsum[1]





%************************************************
\section{Algoritmo parallelizzato}
\label{sec:parallelo}
%************************************************

\lipsum[1]





%************************************************
\section{Risultati}
\label{sec:risultati}
%************************************************


I risultati che abbiamo non sono ancora ottimali per evincere conclusioni precise in merito alle prestazioni ed allo speed-up.
Per ora siamo riusciti a fattorizzare numeri sotto le 50 cifre su un numero massimo di nodi pari a 8, i numeri con una quantità superiore di cifre hanno causato l'uccisione del processo dal gestore di code di Galileo, il supercomputer su cui abbiamo eseguito i calcoli, a tal proposito si vedano gli sviluppo futuri per quest'algoritmo nell'ottica di creare un sistema di checkpoint per non perdere il calcolo fatto qualora si eccedesse il tempo a propria disposizione. \\



\begin{}
#CCQg-n2-30
#time_base time_sieve time_gauss time_totale
0.204440 2.357121 3.806429 6.367990  

#CCQg-n4-30
#time_base time_sieve time_gauss time_totale
0.207865 1.891235 3.797348 5.896448 

#CCQg-n8-30
#time_base time_sieve time_gauss time_totale
0.223254 1.923282 1.788766 3.935302 
\end{}














#Numero fattorizzazioni complete trovate: 540
#time_base time_sieve time_gauss time_totale
0.204832 1.589171 1.309165 3.103168 4 30

#CCQi-n4-40
#Numero fattorizzazioni complete trovate: 2153
#time_base time_sieve time_gauss time_totale
2.937254 194.083372 220.288502 417.309128 4 40

# CCQi-n8-45
#Numero fattorizzazioni complete trovate: 3079
#time_base time_sieve time_gauss time_totale
6.649964 813.154146 609.427356 1429.231466 8 45

#CCQi-n4-45l
#Numero fattorizzazioni complete trovate: 3079
#time_base time_sieve time_gauss time_totale
7.454221 1587.917412 693.236949 2288.608582 4 45

#CCQi-n4-50l
#Numero fattorizzazioni complete trovate: 5640
#time_base time_sieve time_gauss time_totale
29.539861 7530.368473 6922.046000 14481.954334 4 50

# CCQi-n12-50
#timelimit 1

#CCQi-n4-55l
# Jul 12 06:10:54 CEST 2015
# Jul 13 01:31
#Fattorizzazioni per ranks:
#1      2       3       4
#2148   2033    2038    1967
#Numero fattorizzazioni complete trovate: 8186
#killed

# CCQi-n8-55l
# Jul 12 01:28:34 CEST 2015
# Jul 12 12:29
#1164   1044    1081    1037    985     993     956     926
#Numero fattorizzazioni complete trovate: 8186
#killed

#CCQi-n8-60l
walltime 86502 exceeded limit 86400

# CCQi-n12-65l
# walltime 86502 exceeded limit 86400


#CCQg-n2-30
#Numero fattorizzazioni complete trovate: 540
#time_base time_sieve time_gauss time_totale
0.204440 2.357121 3.806429 6.367990  2 30

#CCQg-n4-30
#Numero fattorizzazioni complete trovate: 540
#time_base time_sieve time_gauss time_totale
0.207865 1.891235 3.797348 5.896448 4 30

#CCQg-n8-30
#Numero fattorizzazioni complete trovate: 540
#time_base time_sieve time_gauss time_totale
0.223254 1.923282 1.788766 3.935302 8 30

#CCQg-n2-40
#Numero fattorizzazioni complete trovate: 2153
#time_base time_sieve time_gauss time_totale
3.088096 320.353972 250.406708 573.848776 2 40

#CCQg-n4-40
#Numero fattorizzazioni complete trovate: 2153
#time_base time_sieve time_gauss time_totale
2.855808 202.583528 276.894561 482.333897 4 40

#CCQg-n4-45
#exceeded limit 1800

#CCQg-n8-45
#exceeded limit 1800

#CCQg-n12-45
#Numero fattorizzazioni complete trovate: 3079
#time_base time_sieve time_gauss time_totale
6.327680 641.358716 831.867090 1479.553486 12 45


#CCQg-n4-50l
#Numero fattorizzazioni complete trovate: 5640
#time_base time_sieve time_gauss time_totale
28.530629 7683.857860 5171.678976 12884.067465 4 50


#CCQg-n8-50l
#Numero fattorizzazioni complete trovate: 5640
#time_base time_sieve time_gauss time_totale
25.707317 3956.152933 5621.893790 9603.754040 8 50






3%************************************************
\section{Sviluppi futuri}
\label{sec:sviluppi}
%************************************************


L'algoritmo si presta a diversi sviluppi orientati sia al miglioramento delle prestazioni che al miglioramento dell'interfaccia
con l'utente.

In merito alle prestazioni la prima modifica utile che potrebbe essere apportata è la seguente:
l'architettura del sistema in questo momento è di tipo master slave, nella fattispecie il master svolge semplicemente il compito di suddividere  
i dati e di recuperarli per poi eseguire la parte seriale dell'algoritmo, ossia l'eliminazione Gaussiana.
Sarebbe utile modificare l'algoritmo affinchè nel tempo di attesa anche il processo master eseguisse la parte degli slave così da ottimizzare
l'uso delle cpu.

Considerando l'esecuzione dell'algoritmo in parallelo sui nodi di un supercomputer occorre considerare il problema del walltime, ossia
del tempo massimo assegnato ad un processo dal gestore di code.
Se il processo supera il walltime senza aver terminato correttamente l'esecuzione viene killato.
Questo è un problema piuttosto notevole se si tratta di algoritmi che richiedono diversi giorni di calcolo su determinati dati; occorre quindi
permettere all'algoritmo di crearsi dei checkpoint facendo un salvataggio dei dati intermedi per poi riprendere la computazione dal punto in 
cui era stato bloccato.

/*
        DETTAGLI TECNICI
*/


%\appendix
%\input{paragrafi/Dolor}
% *****************************************************************
% Materiale finale
%******************************************************************
%*******************************************************
% Bibliografia
%*******************************************************
\nocite{*}
\printbibliography
\end{document}
